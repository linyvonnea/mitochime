%   Filename    : chapter_2.tex
% \section*{Chapter 2}\label{sec:researchdesc}
\chapter{Review of Related Literature}
This chapter presents an overview of the literature relevant to the study. It discusses the biological and computational foundations underlying mitochondrial genome analysis and assembly, as well as existing tools, algorithms, and techniques related to chimera detection and genome quality assessment. The chapter aims to highlight the strengths, limitations, and research gaps in current approaches that motivate the development of the present study.

\section{The Mitochondrial Genome}
Mitochondrial genome (mtDNA) is a small, typically circular molecule found in most eukaryotes. It encodes essential genes involved in oxidative phosphorylation and energy metabolism. Because of its conserved structure, mtDNA has become a valuable genetic marker for studies in population genetics and phylogenetics \citep{Anderson1981,Boore1999}. 
In animal species, the mitochondrial genome ranges from 15--20 kilobase and contains 13 protein-coding genes, 22 tRNAs, and two rRNAs arranged compactly without introns \citep{Gray1999}. In comparison to nuclear DNA, the ratio of the number of copies of mtDNA is higher and has simple organization which make it particularly suitable for genome sequencing and assembly studies \citep{Dierckxsens2017}. 


\subsection {Mitochondrial Genome Assembly}
Mitochondrial genome assembly refers to the reconstruction of the complete mitochondrial DNA (mtDNA) sequence from raw or fragmented sequencing reads. It is conducted to obtain high-quality, continuous representations of the mitochondrial genome that can be used for a wide range of analyses, including species identification, phylogenetic reconstruction, evolutionary studies, and investigations of mitochondrial diseases. Because mtDNA evolves rapidly, its assembled sequence provides valuable insights into population structure, lineage divergence, and adaptive evolution across taxa \citep{Boore1999}. Compared to nuclear genome assembly, assembling the mitochondrial genome is often considered more straightforward but still encounters technical challenges such as the formation of chimeric reads. Commonly used tools for mitogenome assembly such as GetOrganelle and MITObim operate under the assumption of organelle genome circularity, and are vulnerable when chimeric reads disrupt this circular structure, resulting in assembly errors \citep{Jin2020, Hahn2013}.

\section{PCR Amplification and Chimera Formation}
PCR plays an important role in NGS library preparation, as it amplifies target DNA fragments for downstream analysis. However as previously mentioned, the amplification process can also introduce chimeric reads which compromises the quality of the input reads supplied to sequencing or assembly workflows. Chimeras typically arise when incomplete extension occurs during a PCR cycle. This causes the DNA polymerase to switch from one template to another and generate hybrid recombinant molecules \citep{Judo1998}. Artificial chimeras are produced through such amplification errors, whereas biological chimeras occur naturally through genomic rearrangements or transcriptional events.

In the context of amplicon-based sequencing, the presence of chimeras can inflate estimates of genetic or microbial diversity and may cause misassemblies during genome reconstruction. \cite{Qin2023} has reported that chimeric sequences may account for more than 10\% of raw reads in amplicon datasets. This artifact tends to be most prominent among rare operational taxonomic units (OTUs) or singletons, which are sometimes misinterpreted as novel diversity, further causing the complication of microbial diversity analyses \citep{Gonzalez2004}. As such, determining and minimizing PCR-induced chimera formation is vital for improving the quality of mitochondrial genome assemblies, and ensuring the reliability of amplicon sequencing data. 

\clearpage

\section{Existing Traditional Approaches for Chimera Detection}
Several computational tools have been developed to identify chimeric sequences in NGS datasets. These tools generally fall into two categories: reference-based and de novo approaches.
Reference-based chimera detection, also known as database-dependent detection, is one of the earliest and most widely used computational strategies for identifying chimeric sequences in amplicon-based community studies. These methods rely on the comparison of each query sequence against a curated, high-quality database of known, non-chimeric reference sequences \citep{Edgar2011}.

On the other hand, the de novo chimera detection, also referred to as reference-free detection, represents an alternative computational paradigm that identifies chimeric sequences without reliance on external reference databases. This method infer chimeras based on internal relationships among the sequences present within the dataset itself, making it particularly advantageous in studies of under explored or taxonomically diverse communities where comprehensive reference databases are unavailable or incomplete \citep{Edgar2011,Edgar2016}. The underlying assumption on this method is that during PCR, true biological sequences are generally more abundant as they are amplified early and dominate the read pool, whereas chimeric sequences appear later and are generally less abundant. The de novo approach leverage this abundance hierarchy, treating the most abundant sequences as supposed parents and testing whether less abundant sequences can be reconstructed as mosaics of these templates. Compositional and structural similarity are also evaluated to check whether different regions of a candidate sequence correspond to distinct high-abundance sequences.

In practice, many modern bioinformatics pipelines combine both paradigms sequentially: an initial de novo step identifies dataset-specific chimeras, followed by a reference-based pass that removes remaining artifacts relative to established databases \citep{Edgar2016}. These two methods of detection form the foundation of tools such as UCHIME and later UCHIME2.

\subsection{UCHIME}
Developed by Edgar et al. \citep{Edgar2011}, UCHIME is one of the most widely used computational tools for detecting chimeric sequences in amplicon sequencing data. The UCHIME algorithm detects chimeras by evaluating how well a query sequence (Q) can be explained as a mosaic of two parent sequences (A and B) from a reference database. The query sequence is first divided into four non-overlapping segments or chunks. Each chunk is independently searched against a reference database that is assumed to be free of chimeras. The best matches to each segment are collected, and from these results, two candidate parent sequences are identified, typically the two sequences that best explain all chunks of the query. Then a three-way alignment among the query (Q) and the two parent candidates (A and B) is done. From this alignment, UCHIME attempts to find a chimeric model (M) which is a hypothetical recombinant sequence formed by concatenating fragments from A and B that best match the observed Q

\subsubsection{Chimeric Alignment and Scoring}  
To decide whether a query is chimeric, UCHIME computes several alignment-based metrics between Q, its top hit (T, the most similar known sequence), and the chimeric model (M). The key differences are measured as: dQT or the number of mismatches between the query and the top hit as well as dQM or the number of mismatches between the query and the chimeric model. From these, a chimera score is calculated to quantify how much better the chimeric model fits the query compared to a single parent. If the model’s similarity to Q exceeds a defined threshold (typically $\ge$0.8\% better identity), the sequence is reported as chimeric. A higher score indicates stronger evidence of chimerism, while lower scores suggest that the sequence is more likely to be authentic.

In de novo mode, UCHIME applies an abundance-driven strategy. Only sequences at least twice as abundant as the query are considered as potential parents. Non-chimeric sequences identified at each step are added iteratively to a growing internal database for subsequent queries.

\subsubsection{Limitations of UCHIME}
Although UCHIME was a significant advancement in chimera detection, it has notable limitations. According to \citep{Edgar2016} and the UCHIME practical notes \citep{EdgarManual}, many of the accuracy results reported in the original 2011 paper were overly optimistic due to unrealistic benchmark designs that assumed complete reference coverage and perfect sequence quality. In practice, UCHIME’s accuracy can decline when (1) the reference database is incomplete or contains erroneous entries; (2) low-divergence chimeras are present, as these closely resemble genuine biological variants; (3) sequence datasets include residual sequencing errors, leading to spurious alignments or misidentification; and (4) the abundance ratio between parent and chimera is distorted by amplification bias. Additionally, UCHIME tends to misclassify sequences as non-chimeric when parent sequences are missing from the database. These limitations motivated the development of UCHIME2.

\subsection{UCHIME2}
To overcome the limitations of its predecessor, UCHIME2 \citep{Edgar2016} introduced several methodological and algorithmic refinements that significantly enhanced the accuracy and reliability of chimera detection. One major improvement lies in its approach to uncertainty handling. In earlier versions, sequences with limited reference support were often incorrectly classified as non-chimeric, increasing the likelihood of false negatives. UCHIME2 addresses this issue by designating such ambiguous sequences as “unknown,” thereby providing a more conservative and reliable classification framework.

Another notable advancement is the introduction of multiple application-specific modes that allow users to tailor the algorithm’s performance to the characteristics of their datasets. The following parameter presets: denoised, balanced, sensitive, specific, and high-confidence, enable researchers to optimize the balance between sensitivity and specificity according to the goals of their analysis.

In comparative evaluations, UCHIME2 demonstrated superior detection performance, achieving sensitivity levels between 93\% and 99\% and lower overall error rates than earlier versions or other contemporary tools such as DECIPHER and ChimeraSlayer. Despite these advances, the study also acknowledged a fundamental limitation in chimera detection: complete error-free identification is theoretically unattainable. This is due to the presence of “perfect fake models,” wherein genuine non-chimeric sequences can be perfectly reconstructed from other reference fragments. This underscore the uncertainty in differentiating authentic biological sequences from artificial recombinants based solely on sequence similarity, emphasizing the need for continued methodological refinement and cautious interpretation of results.

\subsection{CATch}
As previously mentioned, UCHIME \citep{Edgar2011} relied on alignment-based sequences in amplicon data. However, researchers soon observed that different algorithms often produced inconsistent predictions. A sequence might be identified as chimeric by one tool but classified as non-chimeric by another, resulting in unreliable filtering outcomes across studies.

To address these inconsistencies, \cite{Mysara2015} developed the Classifier for Amplicon Tool Chimeras (CATCh), which represents the first ensemble machine learning system designed for chimera detection in 16S rRNA amplicon sequencing. Rather than depending on a single detection strategy, CATCh integrates the outputs of several established tools, including UCHIME, ChimeraSlayer, DECIPHER, Pintail, and Perseus. The individual scores and binary decisions generated by these tools are used as input features for a supervised learning model. The algorithm employs a Support Vector Machine (SVM) with a Pearson VII Universal Kernel (PUK) to determine optimal weightings among the input features and to assign each sequence a probability of being chimeric.

Benchmarking in both reference-based and de novo modes demonstrated significant performance improvements. CATCh achieved sensitivities of approximately 85 percent in reference-based mode and 92 percent in de novo mode, with corresponding specificities of approximately 96 percent and 95 percent. These results indicate that CATCh detected 7 to 12 percent more chimeras than any individual algorithm while maintaining high precision.

\subsection{ChimPipe}
Among the available tools for chimera detection, ChimPipe is a pipeline developed to identify chimeric sequences such as biological chimeras. It uses both discordant paired-end reads and split-read alignments to improve the accuracy and sensitivity of detecting biological chimeras \citep{Rodriguez2017}. By combining these two sources of information, ChimPipe achieves better precision than methods that depend on a single type of indicator.

The pipeline works with many eukaryotic species that have available genome and annotation data \citep{Rodriguez2017}. It can also predict multiple isoforms for each gene pair and identify breakpoint coordinates that are useful for reconstructing and verifying chimeric transcripts. Tests using both simulated and real datasets have shown that ChimPipe maintains high accuracy and reliable performance.

ChimPipe lets users adjust parameters to fit different sequencing protocols or organism characteristics. Experimental results have confirmed that many chimeric transcripts detected by the tool correspond to functional fusion proteins, demonstrating its utility for understanding chimera biology and its potential applications in disease research \citep{Rodriguez2017}.

\section{Machine Learning Approaches for Chimera and Sequence Quality Detection}
Traditional chimera detection tools rely primarily on heuristic or alignment-based rules. Recent advances in machine learning (ML) have demonstrated that models trained on sequence-derived features can effectively capture compositional and structural patterns in biological sequences. Although most existing ML systems such as those used for antibiotic resistance prediction, taxonomic classification, or viral identification are not specifically designed for chimera detection, they highlight how data-driven models can outperform similarity-based heuristics by learning intrinsic sequence signatures. In principle, ML frameworks can integrate indicators such as k-mer frequencies, GC-content variation and split-alignment metrics to identify subtle anomalies that may indicate a chimeric origin \citep{Arango2018,Liang2020,Ren2020}.

\subsection{Feature-Based Representations of Genomic Sequences}
In genomic analysis, feature extraction converts DNA sequences into numerical representations suitable for ML algorithms. A common approach is k-mer frequency analysis, where normalized k-mer counts form the feature vector \citep{Vervier2015}. These features effectively capture local compositional patterns that often differ between authentic and chimeric reads. In particular, deviations in k-mer profiles between adjacent read segments can serve as a compositional signature of template-switching events. Additional descriptors such as GC content and sequence entropy can further distinguish sequence types; in metagenomic classification and virus detection, k-mer-based features have shown strong performance and robustness to noise \citep{Vervier2015,Ren2020}. For chimera detection specifically, abrupt shifts in GC or k-mer composition along a read can indicate junctions between parental fragments. Windowed feature extraction enables models to capture these discontinuities that rule-based algorithms may overlook.

Machine learning models can also leverage alignment-derived features such as the frequency of split alignments, variation in mapping quality, and local coverage irregularities. Split reads and discordant read pairs are classical indicators of genomic junctions and have been formalized in probabilistic frameworks for structural-variant discovery that integrate multiple evidence types \citep{Layer2014}. Similarly, long-read tools such as Sniffles employ split-alignment and coverage anomalies to accurately localize breakpoints \citep{Sedlazeck2018}. Modern aligners such as Minimap2 \citep{Li2018} output supplementary (SA tags) and secondary alignments as well as chaining and alignment-score statistics that can be summarized into quantitative predictors for machine-learning models. These alignment-signal features are particularly relevant to PCR-induced mitochondrial chimeras, where template-switching events produce reads partially matching distinct regions of the same or related genomes. Integrating such cues within a supervised-learning framework enables artifact detection even in datasets lacking complete or perfectly assembled references.

A further biologically grounded descriptor is the length of microhomology at putative junctions. Microhomology refers to short, shared sequences, often in the range of a few to tens of base pairs that are near breakpoints where template-switching events typically happen. Studies of double strand break repair and structural variation have demonstrated that the length of microhomology correlates with the likelihood of microhomology-mediated end joining (MMEJ) or fork-stalled template-switching pathways \citep{Sfeir2015}. In the context of PCR-induced chimeras, template switching during amplification often leaves short identical sequences at the junction of two concatenated fragments. Quantifying the longest exact suffix–prefix overlap at each candidate breakpoint thus provides a mechanistic signature of chimerism and complements both compositional (k-mer) and alignment (SA count) features.

\section{Synthesis of Chimera Detection Approaches}
To provide an integrated overview of the literature discussed in this chapter, Table~\ref{tab:study_summary} summarizes the major chimera detection studies, their methodological approaches, and their known limitations.

\begin{landscape}
\newgeometry{left=1.2cm, right=1.2cm, top=1.8cm, bottom=1.8cm} % Adjust margins for landscape
\begin{table}[ht]
\centering
\caption{Summary of Existing Methods and Research Gaps}
\label{tab:study_summary}
\begin{adjustbox}{max width=1.03\textwidth} % Ensure table fits within page width
\renewcommand{\arraystretch}{1.25} % Increase row height for readability
\begin{tabularx}{1.03\textwidth}{|X|X|X|} % Use tabularx for dynamic column widths
\hline
    \textbf{Method/Study} & \textbf{Scope/Approach} & \textbf{Limitations} \\
\hline
Reference-based Chimera Detection &
Compares query sequences against curated, non-chimeric reference databases; identifies mosaic sequences by evaluating similarity to known templates. &
Depends heavily on completeness and quality of reference databases; often fails when novel taxa or missing parent sequences are present; reduced accuracy for low-divergence chimeras. \\
\hline
De novo Chimera Detection &
Identifies chimeras using only internal dataset relationships; relies on abundance patterns and compositional similarity; reconstructs sequences as mosaics of high-abundance parents. &
Assumes true sequences are more abundant—fails when amplification bias distorts abundance; struggles with evenly abundant parental sequences; can misclassify highly similar true variants. \\
\hline
UCHIME &
Alignment-based chimera detection; segments query sequence, identifies parent candidates, performs 3-way alignment, and computes chimera scores; supports both reference-based and de novo modes. &
Accuracy inflated in original benchmarks; suffers under incomplete databases; poor performance on low-divergence chimeras; sensitive to sequencing errors; misclassifies when parents are missing. \\
\hline
UCHIME2 &
Improved uncertainty handling; classifies ambiguous sequences as unknown; offers multiple sensitivity/specificity modes; more robust with incomplete references; higher sensitivity (93--99\%). &
Cannot achieve perfect accuracy due to ``perfect fake models''; genuine variants may be indistinguishable from artificial recombinants; theoretical detection limit remains. \\
\hline
CATCh &
First ML ensemble tool for 16S chimera detection; integrates outputs of UCHIME, ChimeraSlayer, DECIPHER, Pintail, Perseus via SVM classifier; significantly improves sensitivity and specificity. &
Depends on performance of underlying tools; ML model limited to features they output; ensemble can still misclassify in datasets with extreme novelty or low coverage. \\
\hline
ChimPipe &
Pipeline for detecting fusion genes and transcript-derived chimeras in RNA-seq; uses discordant paired-end reads and split-alignments; predicts isoforms and breakpoint coordinates. &
Designed for RNA-seq, not amplicons; needs high-quality genome and annotation; computationally heavier; limited to organisms with reference genomes. \\
\hline
Machine-Learning Sequence Quality \& Chimera Detection (general) &
Uses k-mer profiles, GC content shifts, entropy, split-read statistics, mapping quality variation, and micro-homology signatures as predictive features; identifies subtle artifacts missed by heuristics. &
Requires labeled training data; model performance depends on feature engineering; may capture dataset-specific biases; limited generalization if training data is narrow or unrepresentative. \\
\hline
\end{tabularx}
\end{adjustbox}
\end{table}
\restoregeometry
\end{landscape}

Across existing studies, no single approach reliably detects all forms of chimeric sequences, particularly those generated by PCR-induced template switching in mitochondrial genomes. Reference-based tools perform poorly when parental sequences are absent; de novo methods rely strongly on abundance assumptions; alignment-based systems show reduced sensitivity to low-divergence chimeras; and ensemble methods inherit the limitations of their component algorithms. RNA-seq–oriented pipelines likewise do not generalize well to organelle data. Although machine learning approaches offer promising feature-based detection, they are rarely applied to mitochondrial genomes and are not trained specifically on PCR-induced organelle chimeras. These limitations indicate a clear research gap: the need for a specialized, feature-driven classifier tailored to mitochondrial PCR-induced chimeras that integrates k-mer composition, split-alignment signals, and micro-homology features to achieve more accurate detection than current heuristic or alignment-based tools.