%   Filename    : chapter_2.tex
% \section*{Chapter 2}\label{sec:researchdesc}
\chapter{Review of Related Literature}
This chapter presents an overview of the literature relevant to the study. It discusses the biological and computational foundations underlying mitochondrial genome analysis and assembly, as well as existing tools, algorithms, and techniques related to chimera detection and genome quality assessment. The chapter aims to highlight the strengths, limitations, and research gaps in current approaches that motivate the development of the present study.

\section{The Mitochondrial Genome}
Mitochondrial genome (mtDNA) is a small, typically circular molecule found in most eukaryotes. It encodes essential genes involved in oxidative phosphorylation and energy metabolism. Because of its conserved structure, mtDNA has become a valuable genetic marker for studies in population genetics and phylogenetics \citep{Anderson1981,Boore1999}. 
In animal species, the mitochondrial genome ranges from 15--20 kilobase and contains 13 protein-coding genes, 22 tRNAs, and two rRNAs arranged compactly without introns \citep{Gray1999}. In comparison to nuclear DNA, the ratio of the number of copies of mtDNA is higher and has simple organization which make it particularly suitable for genome sequencing and assembly studies \citep{Dierckxsens2017}. 


\subsection {Mitochondrial Genome Assembly}
Mitochondrial genome assembly refers to the reconstruction of the complete mitochondrial DNA (mtDNA) sequence from raw or fragmented sequencing reads. It is conducted to obtain high-quality, continuous representations of the mitochondrial genome that can be used for a wide range of analyses, including species identification, phylogenetic reconstruction, evolutionary studies, and investigations of mitochondrial diseases. Because mtDNA evolves rapidly, its assembled sequence provides valuable insights into population structure, lineage divergence, and adaptive evolution across taxa \citep{Boore1999}. Compared to nuclear genome assembly, assembling the mitochondrial genome is often considered more straightforward but still encounters technical challenges such as the formation of chimeric reads. Commonly used tools for mitogenome assembly such as GetOrganelle and MITObim operate under the assumption of organelle genome circularity, and are vulnerable when chimeric reads disrupt this circular structure, resulting in assembly errors \citep{Jin2020, Hahn2013}.

\section{PCR Amplification and Chimera Formation}
PCR plays an important role in NGS library preparation, as it amplifies target DNA fragments for downstream analysis. However as previously mentioned, the amplification process can also introduce chimeric reads which compromises the quality of the input reads supplied to sequencing or assembly workflows. Chimeras typically arise when incomplete extension occurs during a PCR cycle. This causes the DNA polymerase to switch from one template to another and generate hybrid recombinant molecules \citep{Judo1998}. Artificial chimeras are produced through such amplification errors, whereas biological chimeras occur naturally through genomic rearrangements or transcriptional events.

In the context of amplicon-based sequencing, the presence of chimeras can inflate estimates of genetic or microbial diversity and may cause misassemblies during genome reconstruction. \cite{Qin2023} has reported that chimeric sequences may account for more than 10\% of raw reads in amplicon datasets. This artifact tends to be most prominent among rare operational taxonomic units (OTUs) or singletons, which are sometimes misinterpreted as novel diversity, further causing the complication of microbial diversity analyses \citep{Gonzalez2004}. As such, determining and minimizing PCR-induced chimera formation is vital for improving the quality of mitochondrial genome assemblies, and ensuring the reliability of amplicon sequencing data. 

\clearpage

\section{Existing Traditional Approaches for Chimera Detection} \label{sec:abundance_skew}
Several computational tools have been developed to identify chimeric sequences in NGS datasets. These tools generally fall into two categories: reference-based and de novo approaches.
Reference-based chimera detection, also known as database-dependent detection, is one of the earliest and most widely used computational strategies for identifying chimeric sequences in amplicon-based community studies. These methods rely on the comparison of each query sequence against a curated, high-quality database of known, non-chimeric reference sequences \citep{Edgar2011}.

On the other hand, the de novo chimera detection, also referred to as reference-free detection, represents an alternative computational paradigm that identifies chimeric sequences without reliance on external reference databases. This method infer chimeras based on internal relationships among the sequences present within the dataset itself, making it particularly advantageous in studies of under explored or taxonomically diverse communities where comprehensive reference databases are unavailable or incomplete \citep{Edgar2011,Edgar2016}. The underlying assumption on this method is that during PCR, true biological sequences are generally more abundant as they are amplified early and dominate the read pool, whereas chimeric sequences appear later and are generally less abundant. The de novo approach leverage this abundance hierarchy, treating the most abundant sequences as supposed parents and testing whether less abundant sequences can be reconstructed as mosaics of these templates. Compositional and structural similarity are also evaluated to check whether different regions of a candidate sequence correspond to distinct high-abundance sequences.

In practice, many modern bioinformatics pipelines combine both paradigms sequentially: an initial de novo step identifies dataset-specific chimeras, followed by a reference-based pass that removes remaining artifacts relative to established databases \citep{Edgar2016}. These two methods of detection form the foundation of tools such as UCHIME and later UCHIME2.

\subsection{UCHIME}
UCHIME is one of the most widely used tools for detecting chimeric sequences in amplicon-based studies and remains a standard quality-control step in microbial community analysis. Its core strategy is to test whether a query sequence ($Q$) can be explained as a mosaic of two parent sequences, ($A$ and $B$), and to score this relationship using a structured alignment model \citep{Edgar2011}.

In reference mode, UCHIME divides the query into several segments and maps them against a curated database of non-chimeric sequences. Candidate parents are identified, and a three-way alignment is constructed. The algorithm assigns “Yes” votes when different segments of the query match different parents and “No” votes when the alignment contradicts a chimeric pattern. The final score reflects the balance of these votes. In de novo mode, UCHIME operationalizes the abundance-skew principle described earlier: high-abundance sequences are treated as candidate parents, and lower-abundance sequences are evaluated as potential mosaics. This makes the method especially useful when no reliable reference database exists.

Although UCHIME is highly sensitive, it faces key constraints. Chimeras formed from parents with very low divergence (below ~0.8\%) are difficult to detect because they are nearly indistinguishable from sequencing errors. Accuracy in reference mode depends strongly on database completeness, while de novo detection assumes that true parents are both present and sufficiently more abundant, such conditions are not always met.

\subsection{UCHIME2}
UCHIME2 extends the original algorithm with refinements tailored for high-resolution sequencing data. One of its major contributions is a re-evaluation of benchmarking practices. \cite{Edgar2016} demonstrated that earlier accuracy estimates for chimera detection were overly optimistic because they relied on unrealistic scenarios where all true parent sequences were assumed to be present. Using the more rigorous CHSIMA benchmark, UCHIME2 showed the prevalence of “fake models” or real biological sequences that can be perfectly reconstructed as apparent chimeras of other sequences, which suggests that perfect chimera detection is theoretically unattainable. UCHIME2 also introduces several preset modes (e.g., denoised, balanced, sensitive, specific, high-confidence) designed to tune sensitivity and specificity depending on dataset characteristics. These modes allow users to adjust the algorithm to the expected noise level or analytical goals.

Despite these improvements, UCHIME2 must be applied with caution. The author's website manual \citep{EdgarManual} explicitly advises against using UCHIME2 as a standalone chimera-filtering step in OTU clustering or denoising workflows because doing so can inflate both false positives and false negatives.

\subsection{CATch}
As previously mentioned, UCHIME \citep{Edgar2011} relied on alignment-based sequences in amplicon data. However, researchers soon observed that different algorithms often produced inconsistent predictions. A sequence might be identified as chimeric by one tool but classified as non-chimeric by another, resulting in unreliable filtering outcomes across studies.

To address these inconsistencies, \cite{Mysara2015} developed the Classifier for Amplicon Tool Chimeras (CATCh), which represents the first ensemble machine learning system designed for chimera detection in 16S rRNA amplicon sequencing. Rather than depending on a single detection strategy, CATCh integrates the outputs of several established tools, including UCHIME, ChimeraSlayer, DECIPHER, Pintail, and Perseus. The individual scores and binary decisions generated by these tools are used as input features for a supervised learning model. The algorithm employs a Support Vector Machine (SVM) with a Pearson VII Universal Kernel (PUK) to determine optimal weightings among the input features and to assign each sequence a probability of being chimeric.

Benchmarking in both reference-based and de novo modes demonstrated significant performance improvements. CATCh achieved sensitivities of approximately 85 percent in reference-based mode and 92 percent in de novo mode, with corresponding specificities of approximately 96 percent and 95 percent. These results indicate that CATCh detected 7 to 12 percent more chimeras than any individual algorithm while maintaining high precision.

\subsection{ChimPipe}
Among the available tools for chimera detection, ChimPipe is a pipeline developed to identify chimeric sequences such as biological chimeras. It uses both discordant paired-end reads and split-read alignments to improve the accuracy and sensitivity of detecting biological chimeras \citep{Rodriguez2017}. By combining these two sources of information, ChimPipe achieves better precision than methods that depend on a single type of indicator.

The pipeline works with many eukaryotic species that have available genome and annotation data \citep{Rodriguez2017}. It can also predict multiple isoforms for each gene pair and identify breakpoint coordinates that are useful for reconstructing and verifying chimeric transcripts. Tests using both simulated and real datasets have shown that ChimPipe maintains high accuracy and reliable performance.

ChimPipe lets users adjust parameters to fit different sequencing protocols or organism characteristics. Experimental results have confirmed that many chimeric transcripts detected by the tool correspond to functional fusion proteins, demonstrating its utility for understanding chimera biology and its potential applications in disease research \citep{Rodriguez2017}.

\section{Machine Learning Approaches for Chimera and Sequence Quality Detection}
Traditional chimera detection tools rely primarily on heuristic or alignment-based rules. Recent advances in machine learning (ML) have demonstrated that models trained on sequence-derived features can effectively capture compositional and structural patterns in biological sequences. Although most existing ML systems such as those used for antibiotic resistance prediction, taxonomic classification, or viral identification are not specifically designed for chimera detection, they highlight how data-driven models can outperform similarity-based heuristics by learning intrinsic sequence signatures. In principle, ML frameworks can integrate indicators such as k-mer frequencies, GC-content variation and split-alignment metrics to identify subtle anomalies that may indicate a chimeric origin \citep{Arango2018,Liang2020,Ren2020}.

\subsection{Feature-Based Representations of Genomic Sequences}
Feature extraction converts DNA sequences into numerical representations suitable for machine learning models. One approach is k-mer frequency analysis, which counts short nucleotide sequences within a read \citep{Vervier2015}. High-frequency k-mers, including simple repeats such as “AAAAAA,” can highlight repetitive or unusual regions that may occur near chimeric junctions. Comparing k-mer patterns across adjacent parts of a read can help identify such regions, while GC content provides an additional descriptor of local sequence composition \citep{Ren2020}.

Alignment-derived features further inform junction detection. Long-read tools such as Sniffles \citep{Sedlazeck2018} use split alignments to locate breakpoints across extended sequences, whereas short-read aligners like Minimap2 \citep{Li2018} report supplementary and secondary alignments that indicate local discontinuities. Split alignments, where parts of a read map to different regions, can reveal template-switching events. These features complement k-mer profiles and enhance detection of potentially chimeric reads, even in datasets with incomplete references.

Microhomology, or short sequences shared between adjacent segments, is another biologically meaningful feature. Its length, typically a few to tens of base pairs, has been linked to microhomology-mediated repair and template-switching mechanisms \citep{Sfeir2015}. In PCR-induced chimeras, short identical sequences at junctions provide a clear signature of chimerism. Measuring the longest exact overlap at each breakpoint complements k-mer and alignment features and helps identify reads that are potentially chimeric.

\section{Synthesis of Chimera Detection Approaches}
To provide an integrated overview of the literature discussed in this chapter, Table~\ref{tab:study_summary} summarizes the major chimera detection studies, their methodological approaches, and their known limitations.

\newgeometry{top=50pt, bottom=60pt, left=30pt, right=30pt}
\begin{landscape}
\begin{table}[H]
    \centering
    \caption{Comparison of Chimera Detection Approaches and Tools}
    \label{tab:study_summary}

    \begin{tabularx}{\linewidth}{>{\bfseries}p{3.2cm} X X}
        \toprule
        Method / Tool & \textbf{Core Approach} & \textbf{Key Limitations} \\
        \midrule

        Reference-based Detection &
        Compares each query sequence against curated databases of verified, non-chimeric sequences; evaluates segment similarity to identify mosaic patterns. &
        Accuracy depends on database completeness; performs poorly for novel taxa or missing parents; limited sensitivity for low-divergence chimeras. \\
        \addlinespace

        De novo Detection &
        Identifies chimeras using only internal dataset structure; leverages abundance hierarchy and compositional similarity to infer whether low-abundance sequences can be reconstructed from abundant parents. &
        Assumes true sequences are more abundant; fails when amplification bias distorts abundances; struggles when parental sequences are similarly abundant or highly similar. \\
        \addlinespace

        UCHIME &
        Alignment-based model that partitions the query into segments, identifies parent candidates, and computes a chimera score via a three-way alignment; supports reference and de novo modes. &
        Reduced accuracy for very closely related parents ($<$0.8\% divergence); sensitive to incomplete databases; de novo mode fails if parents are absent or not sufficiently more abundant. \\
        \addlinespace

        UCHIME2 &
        Updated UCHIME with improved benchmarking (CHSIMA) and multiple sensitivity/specificity presets; better handles incomplete references and dataset variability. &
        “Fake models’’ limit theoretical accuracy; genuine variants may mimic chimeras; not recommended as a standalone step in OTU or denoising pipelines due to increased false positives/negatives. \\
        \addlinespace

        CATCh &
        First ensemble ML model for 16S chimera detection; integrates outputs of UCHIME, ChimeraSlayer, DECIPHER, Pintail, and Perseus using an SVM to boost overall prediction accuracy. &
        Performance constrained by underlying tools; ML model cannot capture features not present in component algorithms; may misclassify in highly novel or low-coverage datasets. \\
        \addlinespace

        ChimPipe &
        Pipeline for detecting biological chimeras in RNA-seq using discordant paired-end reads and split-read alignments; identifies isoforms and breakpoint coordinates. &
        Requires high-quality genome and annotation; tailored to RNA-seq rather than amplicons; computationally intensive; limited to organisms with available reference genomes. \\
        
        \bottomrule
    \end{tabularx}
\end{table}
\end{landscape}
\restoregeometry


Across existing studies, no single approach reliably detects all forms of chimeric sequences, and the reviewed literature consistently shows that chimeras remain a persistent challenge in genomics and bioinformatics. Although the surveyed tools are not designed specifically for organelle genome assembly, they provide valuable insights into which methodological strategies are effective and where current approaches fall short. These limitations collectively define a clear research gap: the need for a specialized, feature-driven detection framework tailored to PCR-induced mitochondrial chimeras. Addressing this gap aligns with the research objective outlined in Section \ref{subsecsec:researchobjectives}, which is to develop and evaluate a machine learning-based pipeline (MitoChime) that improves the quality of downstream mitochondrial genome assembly. In support of this aim, the subsequent chapters describe the design, implementation, and evaluation of the proposed tool.
