%   Filename    : chapter_2.tex
% \section*{Chapter 2}\label{sec:researchdesc}
\chapter{Review of Related Literature}
This chapter presents an overview of the literature relevant to the study. It discusses the biological and computational foundations underlying mitochondrial genome analysis and assembly, as well as existing tools, algorithms, and techniques related to chimera detection and genome quality assessment. The chapter aims to highlight the strengths, limitations, and research gaps in current approaches that motivate the development of the present study.

\section{The Mitochondrial Genome}
Mitochondrial genome (mtDNA) is a small, typically circular molecule found in most eukaryotes. It encodes essential genes involved in oxidative phosphorylation and energy metabolism. Because of its conserved structure, mtDNA has become a valuable genetic marker for studies in population genetics and phylogenetics \citep{Anderson1981,Boore1999}. 
In animal species, the mitochondrial genome ranges from 15--20 kilobase and contains 13 protein-coding genes, 22 tRNAs, and two rRNAs arranged compactly without introns \citep{Gray1999}. In comparison to nuclear DNA, the ratio of the number of copies of mtDNA is higher and has simple organization which make it particularly suitable for genome sequencing and assembly studies \citep{Dierckxsens2017}. 


\subsection {Mitochondrial Genome Assembly}
Mitochondrial genome assembly refers to the reconstruction of the complete mitochondrial DNA (mtDNA) sequence from raw or fragmented sequencing reads. It is conducted to obtain high-quality, continuous representations of the mitochondrial genome that can be used for a wide range of analyses, including species identification, phylogenetic reconstruction, evolutionary studies, and investigations of mitochondrial diseases. Because mtDNA evolves rapidly, its assembled sequence provides valuable insights into population structure, lineage divergence, and adaptive evolution across taxa \citep{Boore1999}. Compared to nuclear genome assembly, assembling the mitochondrial genome is often considered more straightforward but still encounters technical challenges such as the formation of chimeric reads. Commonly used tools for mitogenome assembly such as GetOrganelle and MITObim operate under the assumption of organelle genome circularity, and are vulnerable when chimeric reads disrupt this circular structure, resulting in assembly errors \citep{Jin2020, Hahn2013}.

\section{PCR Amplification and Chimera Formation}
PCR plays an important role in NGS library preparation, as it amplifies target DNA fragments for downstream analysis. However as previously mentioned, the amplification process can also introduce chimeric reads which compromises the quality of the input reads supplied to sequencing or assembly workflows. Chimeras typically arise when incomplete extension occurs during a PCR cycle. This causes the DNA polymerase to switch from one template to another and generate hybrid recombinant molecules \citep{Judo1998}. Artificial chimeras are produced through such amplification errors, whereas biological chimeras occur naturally through genomic rearrangements or transcriptional events.

In the context of amplicon-based sequencing, the presence of chimeras can inflate estimates of genetic or microbial diversity and may cause misassemblies during genome reconstruction. \cite{Qin2023} has reported that chimeric sequences may account for more than 10\% of raw reads in amplicon datasets. This artifact tends to be most prominent among rare operational taxonomic units (OTUs) or singletons, which are sometimes misinterpreted as novel diversity, further causing the complication of microbial diversity analyses \citep{Gonzalez2004}. As such, determining and minimizing PCR-induced chimera formation is vital for improving the quality of mitochondrial genome assemblies, and ensuring the reliability of amplicon sequencing data. 

\clearpage

\section{Existing Traditional Approaches for Chimera Detection} \label{sec:abundance_skew}
Several computational tools have been developed to identify chimeric sequences in NGS datasets. These tools generally fall into two categories: reference-based and de novo approaches.
Reference-based chimera detection, also known as database-dependent detection, is one of the earliest and most widely used computational strategies for identifying chimeric sequences in amplicon-based community studies. These methods rely on the comparison of each query sequence against a curated, high-quality database of known, non-chimeric reference sequences \citep{Edgar2011}.

On the other hand, the de novo chimera detection, also referred to as reference-free detection, represents an alternative computational paradigm that identifies chimeric sequences without reliance on external reference databases. This method infer chimeras based on internal relationships among the sequences present within the dataset itself, making it particularly advantageous in studies of under explored or taxonomically diverse communities where comprehensive reference databases are unavailable or incomplete \citep{Edgar2011,Edgar2016}. The underlying assumption on this method is that during PCR, true biological sequences are generally more abundant as they are amplified early and dominate the read pool, whereas chimeric sequences appear later and are generally less abundant. The de novo approach leverage this abundance hierarchy, treating the most abundant sequences as supposed parents and testing whether less abundant sequences can be reconstructed as mosaics of these templates. Compositional and structural similarity are also evaluated to check whether different regions of a candidate sequence correspond to distinct high-abundance sequences.

In practice, many modern bioinformatics pipelines combine both paradigms sequentially: an initial de novo step identifies dataset-specific chimeras, followed by a reference-based pass that removes remaining artifacts relative to established databases \citep{Edgar2016}. These two methods of detection form the foundation of tools such as UCHIME and later UCHIME2.

\subsection{UCHIME}
UCHIME is one of the most widely used computational tools for detecting chimeric sequences in amplicon sequencing data, as it serves as a critical quality control step to prevent the misinterpretation of PCR artifacts as novel biological diversity. The algorithm operates by searching for a model ($M$) where a query ($Q$) sequence can be perfectly explained as a combination of two parent sequences, denoted as $A$ and $B$ \citep{Edgar2011}.

In reference mode, UCHIME divides the query into four chunks and maps them to a trusted chimeric-free database to identify candidate parents. It then constructs a three-way alignment to calculate a score based on ``votes.'' A ``Yes'' vote indicates the query aligns with parent $A$ in one region and parent $B$ in another, while a ``No'' vote penalizes the score if the query diverges from the expected chimeric model. In de novo mode, the algorithm operationalizes the abundance skew principle described in Section~\ref{sec:abundance_skew}. Instead of using an external database, UCHIME dynamically treats the sample’s own high-abundance sequences as a reference database, testing if lower-abundance sequences can be reconstructed as mosaics of these internal ancestors \citep{Edgar2011}.

Despite its high sensitivity, UCHIME has inherent limitations rooted in sequence divergence and database quality. The algorithm struggles to detect chimeras formed from parents that are very closely related, specifically when the sequence divergence between parents is less than roughly 0.8\%, as the signal-to-noise ratio becomes too low to distinguish a crossover event from sequencing error \citep{Edgar2011}. Furthermore, in reference mode, the accuracy is strictly bound by the completeness of the database; if true parents are absent, the tool may fail to identify the chimera or produce false positives. Similarly, the de novo mode relies on the assumption that parents are present and sufficiently more abundant in the sample, which may not hold true in unevenly amplified samples or complex communities.

\subsection{UCHIME2}
Building upon the original algorithm, UCHIME2 was developed to address the nuances of high-resolution amplicon sequencing. A key contribution of the UCHIME2 study was the critical re-evaluation of chimera detection benchmarks. In the UCHIME2 paper \citep{Edgar2016} and the UCHIME in practice website \citep{EdgarManual}, the author has noted that the accuracy results reported in the original UCHIME paper were ``highly over-optimistic'' because they relied on unrealistic benchmark designs where parent sequences were assumed to be 100\% known and present. UCHIME2 introduced more rigorous testing (the CHSIMA benchmark), revealing that ``fake models,'' where a valid biological sequence perfectly mimics a chimera of two other valid sequences, are far more common than previously assumed. This discovery suggests that error-free detection is impossible in principle \citep{Edgar2016}.Another notable improvement is the introduction of multiple application-specific modes that allow users to tailor the algorithm’s performance to the characteristics of their datasets. The following parameter presets: denoised, balanced, sensitive, specific, and high-confidence, enable researchers to optimize the balance between sensitivity and specificity according to the goals of their analysis.

However despite these advancements, the practical application of UCHIME2 requires caution. The author explicitly advises against using UCHIME2 as a stand-alone tool in standard OTU clustering or denoising pipelines. Using UCHIME2 as an independent filtering step in these workflows is discouraged, as it often results in significantly higher error rates, increasing both false positives (discarding valid sequences) and false negatives (retaining chimeras) \citep{Edgar2016}.

\subsection{CATch}
As previously mentioned, UCHIME \citep{Edgar2011} relied on alignment-based sequences in amplicon data. However, researchers soon observed that different algorithms often produced inconsistent predictions. A sequence might be identified as chimeric by one tool but classified as non-chimeric by another, resulting in unreliable filtering outcomes across studies.

To address these inconsistencies, \cite{Mysara2015} developed the Classifier for Amplicon Tool Chimeras (CATCh), which represents the first ensemble machine learning system designed for chimera detection in 16S rRNA amplicon sequencing. Rather than depending on a single detection strategy, CATCh integrates the outputs of several established tools, including UCHIME, ChimeraSlayer, DECIPHER, Pintail, and Perseus. The individual scores and binary decisions generated by these tools are used as input features for a supervised learning model. The algorithm employs a Support Vector Machine (SVM) with a Pearson VII Universal Kernel (PUK) to determine optimal weightings among the input features and to assign each sequence a probability of being chimeric.

Benchmarking in both reference-based and de novo modes demonstrated significant performance improvements. CATCh achieved sensitivities of approximately 85 percent in reference-based mode and 92 percent in de novo mode, with corresponding specificities of approximately 96 percent and 95 percent. These results indicate that CATCh detected 7 to 12 percent more chimeras than any individual algorithm while maintaining high precision.

\subsection{ChimPipe}
Among the available tools for chimera detection, ChimPipe is a pipeline developed to identify chimeric sequences such as biological chimeras. It uses both discordant paired-end reads and split-read alignments to improve the accuracy and sensitivity of detecting biological chimeras \citep{Rodriguez2017}. By combining these two sources of information, ChimPipe achieves better precision than methods that depend on a single type of indicator.

The pipeline works with many eukaryotic species that have available genome and annotation data \citep{Rodriguez2017}. It can also predict multiple isoforms for each gene pair and identify breakpoint coordinates that are useful for reconstructing and verifying chimeric transcripts. Tests using both simulated and real datasets have shown that ChimPipe maintains high accuracy and reliable performance.

ChimPipe lets users adjust parameters to fit different sequencing protocols or organism characteristics. Experimental results have confirmed that many chimeric transcripts detected by the tool correspond to functional fusion proteins, demonstrating its utility for understanding chimera biology and its potential applications in disease research \citep{Rodriguez2017}.

\section{Machine Learning Approaches for Chimera and Sequence Quality Detection}
Traditional chimera detection tools rely primarily on heuristic or alignment-based rules. Recent advances in machine learning (ML) have demonstrated that models trained on sequence-derived features can effectively capture compositional and structural patterns in biological sequences. Although most existing ML systems such as those used for antibiotic resistance prediction, taxonomic classification, or viral identification are not specifically designed for chimera detection, they highlight how data-driven models can outperform similarity-based heuristics by learning intrinsic sequence signatures. In principle, ML frameworks can integrate indicators such as k-mer frequencies, GC-content variation and split-alignment metrics to identify subtle anomalies that may indicate a chimeric origin \citep{Arango2018,Liang2020,Ren2020}.

\subsection{Feature-Based Representations of Genomic Sequences}
Feature extraction converts DNA sequences into numerical representations suitable for machine-learning models. One approach is k-mer frequency analysis, which counts short nucleotide sequences within a read \citep{Vervier2015}. High-frequency k-mers, including simple repeats such as “AAAAAA,” can highlight repetitive or unusual regions that may occur near chimeric junctions. Comparing k-mer patterns across adjacent parts of a read can help identify such regions, while GC content provides an additional descriptor of local sequence composition \citep{Ren2020}.

Alignment-derived features further inform junction detection. Long-read tools such as Sniffles \citep{Sedlazeck2018} use split alignments to locate breakpoints across extended sequences, whereas short-read aligners like Minimap2 \citep{Li2018} report supplementary and secondary alignments that indicate local discontinuities. Split alignments, where parts of a read map to different regions, can reveal template-switching events. These features complement k-mer profiles and enhance detection of potentially chimeric reads, even in datasets with incomplete references.

Microhomology, or short sequences shared between adjacent segments, is another biologically meaningful feature. Its length, typically a few to tens of base pairs, has been linked to microhomology-mediated repair and template-switching mechanisms \citep{Sfeir2015}. In PCR-induced chimeras, short identical sequences at junctions provide a clear signature of chimerism. Measuring the longest exact overlap at each breakpoint complements k-mer and alignment features and helps identify reads that are potentially chimeric.

\section{Synthesis of Chimera Detection Approaches}
To provide an integrated overview of the literature discussed in this chapter, Table~\ref{tab:study_summary} summarizes the major chimera detection studies, their methodological approaches, and their known limitations.

\newgeometry{top=50pt, bottom=60pt, left=30pt, right=30pt}
\begin{landscape}
    \begin{table}[H]
        \centering
        \caption{Comparison of Chimera Detection Methods}
        \label{tab:study_summary}
        
        % Column setup:
        % l = left align (adjust width as needed, e.g., p{3cm})
        % X = auto-width column that wraps text
        \begin{tabularx}{\linewidth}{>{\bfseries}p{3cm} X X}
            \toprule
            Methods & \textbf{Approach} & \textbf{Limitations} \\
            \midrule
            
            Reference-based Chimera Detection &
            Compares query sequences against curated, non-chimeric reference databases; identifies mosaic sequences by evaluating similarity to known templates. &
            Depends heavily on completeness and quality of reference databases; often fails when novel taxa or missing parent sequences are present; reduced accuracy for low-divergence chimeras. \\
            \addlinespace % Adds a little breathing room between rows
            
            De novo Chimera Detection &
            Identifies chimeras using only internal dataset relationships; relies on abundance patterns and compositional similarity; reconstructs sequences as mosaics of high-abundance parents. &
            Assumes true sequences are more abundant—fails when amplification bias distorts abundance; struggles with evenly abundant parental sequences; can misclassify highly similar true variants. \\
            \addlinespace
            
            UCHIME &
            Alignment-based chimera detection; segments query sequence, identifies parent candidates, performs 3-way alignment, and computes chimera scores; supports both reference-based and de novo modes. &
            Accuracy inflated in original benchmarks; suffers under incomplete databases; poor performance on low-divergence chimeras; sensitive to sequencing errors; misclassifies when parents are missing. \\
            \addlinespace
            
            UCHIME2 &
            Improved initial UCHIME benchmarking; offers multiple sensitivity/specificity modes; more robust with incomplete references; higher sensitivity. &
            Cannot achieve perfect accuracy due to ``perfect fake models''; genuine variants may be indistinguishable from artificial recombinants; theoretical detection limit remains. \\
            \addlinespace
            
            CATCh &
            First ML ensemble tool for 16S chimera detection; integrates outputs of UCHIME, ChimeraSlayer, DECIPHER, Pintail, Perseus via SVM classifier; significantly improves sensitivity and specificity. &
            Depends on performance of underlying tools; ML model limited to features they output; ensemble can still misclassify in datasets with extreme novelty or low coverage. \\
            \addlinespace
            
            ChimPipe &
            Pipeline for detecting fusion genes and transcript-derived chimeras in RNA-seq; uses discordant paired-end reads and split-alignments; predicts isoforms and breakpoint coordinates. &
            Designed for RNA-seq, not amplicons; needs high-quality genome and annotation; computationally heavier; limited to organisms with reference genomes. \\
            
            \bottomrule
        \end{tabularx}
    \end{table}
\end{landscape}
\restoregeometry

Across existing studies, no single approach reliably detects all forms of chimeric sequences, particularly those generated by PCR-induced template switching in mitochondrial genomes. Reference-based tools perform poorly when parental sequences are absent; de novo methods rely strongly on abundance assumptions; alignment-based systems show reduced sensitivity to low-divergence chimeras; and ensemble methods inherit the limitations of their component algorithms. RNA-seq–oriented pipelines likewise do not generalize well to organelle data. Although machine learning approaches offer promising feature-based detection, they are rarely applied to mitochondrial genomes and are not trained specifically on PCR-induced organelle chimeras. These limitations indicate a clear research gap: the need for a specialized, feature-driven classifier tailored to mitochondrial PCR-induced chimeras that integrates k-mer composition, split-alignment signals, and microhomology features to achieve more accurate detection than current heuristic or alignment-based tools.