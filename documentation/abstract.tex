% %   Filename    : abstract.tex 
% \begin{abstract}

% Currently, little work is being done on the development of digital ontologies, particularly that of the folklore of Western Visayas. However, there exists a digital ontology developed by \citeA{dimzon2015} which stores various Western Visayan oral traditions, including folk narratives. To fill this digital preservation gap, the researchers enhanced and expanded the original ontology to accompany more depth of information and store more folk narratives from Panay Island, specifically myths, legends, and folk tales. In addition, the researchers developed a chatbot capable of providing insights and details on the stored Panayanon folk narratives. Specifically, the researchers to created a knowledge base of Panayanon folk narratives and subsequently developed and trained a chatbot to understand and answer inquiries about the Panayanon folk narratives.

% \begin{flushleft}
% \begin{tabular}{lp{3in}}
% \hspace{-0.5em}\textbf{Keywords:}\hspace{0.25em} &
% Philippine folk literature, 
% Digital preservation,
% Ontology-based system, 
% Chat bot
% \\
% \end{tabular}
% \end{flushleft}
% \end{abstract}
